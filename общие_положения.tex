\subsection{Основание для разработки}

\subsubsection{}
Импульсная нейроморфная сверхбольшая интегральная схема (СБИС) <<MNTABP3>>
(в дальнейшем, MNTABP3) и
комплект разработчика программного обеспечения~(в дальнейшем, КРПО)
разрабатывается в рамках
НИОКР <<...>>.

\subsubsection{}
Основанием для разработки является Договор от ... 2024~гoда №~....

\subsubsection{}
Организация, выполняющая разработку СБИС <<MNTABP3>> и КРПО: ООО <<Мотив~НТ>>.

\subsubsection{}
Срок проведения разработки: от ... по ....

\subsection{Цель и задачи разработки}

\subsubsection{}
Импульсная нейроморфная СБИС <<MNTABP3>> в рамках НИОКР <<...>> предназначена
для обработки аудио-потоков данных, а так же данных от других источников
сигналов с целью их фильтрации и выделения полезной информации. Пославленную
задачу разрабатываемая СБИС будет выполнять с применением импульсных
нейросетевых алгоритмов. Для осуществления обозначенной функции MNTABP3
устанавливается как компонент в разрабатываемый в НИОКР модуль или подключается
к нему с помощью специального интерфейса.

\subsubsection{}
Комплект разработчика программного обеспечения предназначен для подготовки
нейронных сетей, компиляции не нейросетевых алгоритмов и укладки их в вычислительную
сеть, образованную одной или несколькими СБИС <<MNTABP3>>.

\subsubsection{}
Целью разработки СБИС <<MNTABP3>> ялвляется улучшение качества и эффективности
обработки аудио и других потоков данных и предоставление возможности классификации
и распознавания различных артефактов в этих потоках данных с помощью
персперктивных энергоэффективных испульсных нейроморфных алгоритмов обрабоки.
Цель достигается за счёт использования новых архитектурных подходов при
создании СБИС, которые на аппаратном уровне наиболее эффективным образом
позволяют реализовывать означенные алгоритмы.

\subsubsection{}
В процессе разработки будут решены следующие задачи:
\begin{itemize}
\item создана отечественная импульсная нейроморфная СБИС, предназначенная
  для аппаратного ускорения исполнения различных нейросетевых алгоритмов, а так
  же различных не нейросетевых задач
  с использованием новых энергоэффективных архитектурных решений;
\item создан КРПО для реализаации на MNTABP3 различных
  задач, позволяющее применять СБИС для решения широкого круга задач как в рамках,
  так и за рамками НИОКР <<...>>.
\end{itemize}

\subsection{Особенности архитектурных решений в СБИС <<MNTABP3>>}

\subsubsection{}
Полностью цифровая реализация.
Ряд известных импульсных нейроморфных вычислителей являются
аналого-цифровыми. В таких архитектурах нейроны моделируются аналоговыми схемными
решениями, а веса синапсов и коммуникации между нейронами выполняются на основе
цифровых схем.
При таком подходе, обещают достигнуть более высокой плотности упаковки
элементов (нейронов и синапсов) и более низкого энергопотребления, однако,
схемы считывания и кондиционирования сигналов с таких аналоговых элементов
оказываются громоздкими и далеко не малопотребляющими. Аналоговые узлы
интегральных схем сильно чувствительны к особенностям производственного
процесса, к температурным режимам и различным шумам, аналоговые узлы
интегральных схем требуют более трудоемкого, почти ручного процесса
проектирования, и для современных субмикронных технологий, аналоговые блоки
занимают площадь на порядки большую, чем цифровые решения. К тому же, есть
большая сложность <<привязки>> параметров программных моделей импульсных
нейронных сетей к аналоговым схемам. В данном проекте предлагается полностью
цифровая реализация на основе современной монолитной кремниевой технологии,
которая обеспечивается высокую компактность вычислительных узлов и
детерминированное поведение СБИС вне зависимости от экземпляра микросхемы.

\subsubsection{}
Память и обработка интегрированы. В предлагаемом проекте мы отказываемся от
архитектуры фон Неймана. Будет применена технология <<вычисления в памяти>>.
В этом случае декодер инструкций, арифметико-логические устройства (АЛУ) и
другие вспомогательные узлы упрощены и располагаются в непосредственной
близости с необходимым блоками памяти. Блоки памяти могут быть разделены по ролям и
иметь необходимое количество интерфейсов до такого уровня, что нет необходимости
в организации глубокого кэширования и передачи данных пакетами. Тем самым
будет устранена проблема <<узкого горлышка>> традиционных вычислительных
архитектур --- большое количество пересылок данных между процессором и памятью
по общей шине, работа такого интерфейса существенно ограничивается общую
производительность системы, её масштабируемость и приводит к повышенному
энергопотреблению.

\subsubsection{}
Многоядерная реализация. Предлагаемая в проекте архитектура представляет из
себя двумерную решетку ядер. Каждое ядро может моделировать ограниченное количество
нейронов, в случае реализации нейросетевого алгоритма. Ядро является синхронным,
но для синхронизации и обмена данными между ядрами используется механизм
пересылки сообщений, которые могут выступать как в роли события, так и
в роли носителя скалярной величины.
Каждое ядро содержит блок локальной памяти, в которой во время работы хранятся
параметры моделируемых нейронов и информация о связях моделируемых нейронов, либо
локальные данные при выполнении не нейросетевых массовопараллельных алгоритмов.

\subsubsection{}
Сравнительно простая реализация ядра. Реализация технологии
<<вычисление в памяти>> упрощает конструкцию ядра по сравнению с современными
ядрами с архитектурой фон Неймана. Крайняя форма упрощения модели нейрона
до конфигурируемого конечного автомата применяется в таких СБИС, как
TrueNorth~\cite{nasic:truenorth},
Intel Loihi~\cite{nasic:loihi},
Tianjic~\cite{nasic:tianjic}
и <<Алтай-1>>~\cite{nasic:altai1}. Однако этого оказывается недостаточным в том
случае, если необходимо реализовать <<онлайн>> обучение или внести в функцию
нейрона заранее не предусмотренные свойства, а также при исполнении
массовопараллельных не нейросетевых алгоритмов. Второй вид упрощения
ядра --- отказ от операции умножения с переходом
к сообщениям-спайкам, которые передают только события. Такое решение
упрощает ядро, однако, на сегодня известно, что элементы искусственной
нейронной сети в тесной связи с импульсной значительно увеличивает
эффективность исполнения задачи. Выше изложенные соображение привели нас к
такой структуре ядра, в котором вместо конечного
автомата используется программируемый автомат и добавлена операция умножения.

\subsubsection{}
Масштабируемость. Предлагаемая архитектура обладает конструктивно
неограниченной масштабируемостью. Используемая для маршрутизации сообщений
относительная адресация целевого ядра позволяет масштабировать систему до
размеров, которые ограничены только требованиями по энергопотреблению и
массогабаритными параметрам.

\subsubsection{}
Отказоустойчивость. К достоинствам предлагаемой архитектуры можно
отнести высокую отказоустойчивость. При отказе части ядер система в целом
остается работоспособной. В алгоритмы маршрутизации заложены возможности
достижения ядер назначения по обходным маршрутам. Отказоустойчивость,
заложенная в архитектуру, снижает требования к дефектности производства
и к условиям эксплуатации.

